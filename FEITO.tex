\chapter{Como foi feito este livro}

Em 2002, a linguista Patience Epps morou com os Hupd’äh da aldeia de Barreira Alta para fazer uma pesquisa sobre a língua Hup. Um dia, ela pediu para o senhor Mario Andrade Pires, um sábio ancião hup, contar para ela a história dos cantos do homem-sombra.

Patience ouviu atentamente a história, gravou-a com seu gravador, escreveu em seu caderno e depois, com a ajuda dos professores Hupd’äh, traduziu a narrativa para o português. Ela pediu também para outras pessoas contarem histórias antigas dos Hupd’äh.

Com estas narrativas ela preparou o livro \textit{Hupd’äh n\ii{}h P\ii{}n\II{}gd’äh -- Contos
dos Hupd’äh}, que até hoje ajuda os alunos das escolas Hupd’äh a aprenderem a ler e a escrever em sua língua.

A Editora Hedra interessou-se em começar a publicar essas histórias para as crianças Hupd’äh e para as crianças brasileiras, numa edição infantil bilíngue. O livro \textit{Bat\II{}\,b Yám P\ii n\II g -- Cantos do homem-sombra} traz a primeira história desse livro maior de contos
Hupd’äh, enriquecidos pelas belas ilustrações de Anita Ekman.

Para preparar o livro, Patience contou com a ajuda do antropólogo Danilo Paiva Ramos, que também morou com os Hupd’äh para fazer sua pesquisa.

O canto Way Naku (o canto do homem-sombra), que também fez parte desse livro, foi cantado
para o Danilo pelo senhor Ponciano Salustiano Ramos, da aldeia de Tat Dëh, em 2010.


