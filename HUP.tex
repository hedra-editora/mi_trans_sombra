\chapter{Para ler as palavras hup}

Para a grafia em geral dos termos da língua hup, adotou"-se como
referência o dicionário de língua hup elaborado pelo linguista Henri
Ramirez, \textit{A língua dos Hupd'äh do Alto Rio Negro}.\footnote{Associação Saúde
Sem Limites. São Paulo: 2006.} Seguindo Ramirez, mantém-se a acentuação
das vogais de acordo com a nasalidade --- indicada por um \textit{til} --- e o tom --- indicado por um acento grave agudo ou grave.

\section{o alfabeto}

Ramirez propõe que o alfabeto hup tem 25 letras: \textit{a}, \textit{ä}, \textit{b}, \textit{ç}, \textit{e}, \textit{ë}, \textit{g},
\textit{h}, \textit{i}, \textit{ɨ}, \textit{j}, \textit{k}, \textit{m}, \textit{n}, \textit{o}, \textit{ö}, \textit{p}, \textit{r}, \textit{s}, \textit{t}, \textit{u}, \textit{w}, y e '.\footnote{Oclusão
glotal.} Destas, 16 são consoantes, nove são vogais e 11 são consoantes
laringalizadas: \textit{b'}, \textit{d'}, \textit{r'}, \textit{j'}, \textit{g'}, \textit{m'}, \textit{n'}, \textit{w'}, \textit{y'}, \textit{s'}, \textit{k'}.

