\title{Os cantos do homem-sombra}

\title{Batib’ yám pínig}



Yuyu-Dëh Mário Pires e Tat-Dëh Ponciano Socot (\textit{narração})

Patience Epps e Danilo Paiva Ramos (\textit{organização})

Anita Ekman (\textit{ilustração})

2ª edição

\chapter{}

\textbf{edição brasileira©} Hedra 2022\\
\textbf{organização} Patience Epps e Danilo Paiva Ramos\\
\textbf{ilustração©} Anita Ekman

\textbf{coordenação da coleção} Luísa Valentini\\
\textbf{edição} Jorge Sallum\\
\textbf{coedição} Suzana Salama\\
\textbf{assistência editorial} Paulo Henrique Pompermaier\\
\textbf{revisão} Renier Silva\\
\textbf{capa} Lucas Kroëff\\

\textbf{\textsc{isbn}} 978-65-89705-73-4

\textbf{conselho editorial} Adriano Scatolin, Antonio Valverde, Caio Gagliardi, Jorge Sallum, Ricardo Valle, Tales Ab'Saber, Tâmis Parron
 
\bigskip
\textit{Grafia atualizada segundo o Acordo Ortográfico da Língua
Portuguesa de 1990, em vigor no Brasil desde 2009.}\\

\vfill
\textit{Direitos reservados em língua\\ 
portuguesa somente para o Brasil}

\textsc{editora hedra ltda.}\\
R.~Fradique Coutinho, 1139 (subsolo)\\
05416--011 São Paulo \textsc{sp} Brasil\\
Telefone/Fax +55 11 3097 8304

editora@hedra.com.br\\
www.hedra.com.br

Foi feito o depósito legal.

\chapter{}

\textbf{Os cantos do homem-sombra} é a história do encontro de um Hup com um \textit{homem-sombra} chamado Way Naku. Ela faz parte de uma série de histórias dos Hupd'äh --- que vivem em aldeias espalhadas pela floresta amazônica, na região do Alto Rio Negro, fronteira entre o Brasil e a Colômbia --- sobre a \textit{gente-sombra}. Os \textit{homens} e \textit{mulheres-sombra} são muito perigosos e usam roupas coloridas, além de caçar e fazer mal aos Hup. Uma dessas roupas tem cor de sombra, daí seu nome. A \textit{gente-sombra} causa doenças e pode até matar. Eles comem a carne e o espírito dos humanos. Mas muitos deles são sábios e conhecem cantos, mitos e benzimentos. 

\textbf{Patience Epps} ​é linguista, professora da Universidade do Texas \textsc{(ut)} e trabalha com os Hupd'äh desde 2000.

\textbf{Danilo Paiva Ramos} é antropólogo e professor adjunto do Departamento de Antropologia e Etnologia da Universidade Federal da Bahia \textsc{(ufba)}. Além das pesquisas etnográficas, com ênfase em xamanismo, discurso, performance, vida ritual e territorialidade, é engajado em causas ligadas aos direitos indígenas. Assessora também o povo Hupd'äh para a construção de seus Planos de Gestão Territorial e Ambiental (\textsc{pgta--hup}).

\textbf{Anita Ekman} é artista visual, \textit{performer} e ilustradora que trabalha com as artes ameríndias e afro-brasileiras. Como especialista em arte indígena, trabalhou na formação da coleção \textit{Great Masters of Popular Art in Ibero-America}, do Banamex Cultural Fund.

\textbf{Mundo Indígena} reúne materiais produzidos com pensadores de diferentes povos indígenas e pessoas que pesquisam, trabalham ou lutam pela garantia de seus direitos. Os livros foram feitos para serem utilizados pelas comunidades envolvidas na sua produção, e por isso uma parte significativa das obras é bilíngue. Esperamos divulgar a imensa diversidade linguística dos povos indígenas no Brasil, que compreende mais de 150 línguas pertencentes a mais de trinta famílias linguísticas.


\chapter{Apresentação}

Os Hupd’äh são um povo indígena que vive em aldeias espalhadas pela floresta Amazônica na região do Alto Rio Negro, no estado do Amazonas, na fronteira com a Colômbia.

A população total é cerca de 1.500 pessoas, que moram em 35 aldeias diferentes. Antigamente, moravam de 15 a 50 pessoas nas aldeias Hupd’äh, mas atualmente as comunidades estão maiores e podem abrigar até 200 pessoas.

Os Hupd’äh costumam caçar com arco e flecha e com zarabatana. Pescam diferentes tipos de peixes nos igarapés, riachos que correm no interior da mata.

As mulheres dedicam-se diariamente às roças de maniva, um tipo de mandioca, mas também ajudam os homens na caça, na pesca e na coleta de frutas.

Quando viajam para visitar parentes em outras aldeias, os Hupd’äh andam por longos caminhos e acampam à beira do igarapés para descansar. Nestas viagens, é comum que, além de encontrarem animais como pacas, tamanduás, antas e cobras, os Hupd’äh encontrem outros tipos de gente, como a \textit{gente-sombra}, \textit{gente-onça} e a \textit{gente-árvore}. Por isso, é preciso ter cuidado e respeitar essas outras gentes, para que elas não façam mal aos viajantes.

\section{a língua}

A língua hup é a primeira a ser falada pelas crianças Hupd’äh. Esta é uma língua tonal, muito
diferente do português. A língua hup é semelhante às línguas dos povos Yuhupdëh,
Dâw e Nadëb (Kuyawi) que também vivem em comunidades na região do Alto Rio Negro.

Desde 2001, os professores Hupd’äh começaram a participar de cursos para melhorar as
escolas de suas aldeias. Nesses cursos, os professores Hupd’äh tiveram a ajuda de linguistas e antropólogos para descrever sua língua, criar um modo de escrevê-la e compreender os princípios gramaticais.

Escutar as histórias dos anciões, gravá-las, escrevê-las e reproduzi-las permite que
crianças e adultos aprendam a ler e escrever na língua hupd’äh.

Também durante o período foi elaborado um dicionário hupd’äh/\,português, em parceria
com pesquisadores de diferentes universidades, além de cartilhas da língua hup.

\chapter{Como foi feito este livro}

Em 2002, a linguista Patience Epps morou com os Hupd’äh da aldeia de Barreira Alta para fazer uma pesquisa sobre a língua Hup. Um dia, ela pediu para o senhor Mario Andrade Pires, um sábio ancião hup, contar para ela a história dos \textit{cantos do homem-sombra}.

Patience ouviu atentamente a história, gravou-a com seu gravador, escreveu em seu caderno e depois, com a ajuda dos professores hupd’äh, traduziu a narrativa para o português. Ela pediu também para outras pessoas contarem histórias antigas dos Hupd’äh.

Com estas narrativas ela preparou o livro \textit{Hupd’äh nɨh Pɨnɨ̗gd’äh}, ou os \textit{Contos dos Hupd’äh}, que até hoje ajuda os alunos das escolas Hupd’äh a aprenderem a ler e a escrever em sua língua.

A Hedra interessou-se em publicar essas histórias, numa edição bilíngue. O livro \textit{Batɨ̗b  Yám Pɨnɨ̗g}, \textit{Os cantos do homem-sombra}, traz a primeira história desse livro maior de contos Hupd’äh, enriquecida pelas belas ilustrações de Anita Ekman. Para preparar o livro, Patience contou com a ajuda do antropólogo Danilo Paiva Ramos, que também morou com os Hupd’äh para fazer sua pesquisa.

O canto \textit{Way Naku}, do homem-sombra, que também fez parte desse livro, foi cantado
para o Danilo pelo senhor Ponciano Salustiano Ramos, da aldeia de Tat Dëh, em 2010.

\chapter{Para ler as palavras hup}

Para a grafia em geral dos termos da língua hup, adotou"-se como
referência o dicionário de língua hup elaborado pelo linguista Henri
Ramirez, \textit{A língua dos Hupd'äh do Alto Rio Negro} (Associação Saúde
Sem Limites, 2006). Seguindo Ramirez, mantém-se a acentuação
das vogais de acordo com a nasalidade --- indicada por um \textit{til} --- e o tom --- indicado por um acento grave agudo ou grave.

\section{o alfabeto}

Ramirez propõe que o alfabeto hup tem 25 letras: \textit{a}, \textit{ä}, \textit{b}, \textit{ç}, \textit{e}, \textit{ë}, \textit{g},
\textit{h}, \textit{i}, \textit{ɨ}, \textit{j}, \textit{k}, \textit{m}, \textit{n}, \textit{o}, \textit{ö}, \textit{p}, \textit{r}, \textit{s}, \textit{t}, \textit{u}, \textit{w}, y e '.\footnote{Oclusão
glotal.} Destas, 16 são consoantes, nove são vogais e 11 são consoantes
laringalizadas: \textit{b'}, \textit{d'}, \textit{r'}, \textit{j'}, \textit{g'}, \textit{m'}, \textit{n'}, \textit{w'}, \textit{y'}, \textit{s'}, \textit{k'}.


\chapter{Os cantos do homem-sombra}

\textit{Os Hupd'äh contam muitas histórias sobre pessoas que, andando pela mata, se
encontraram com seres da gente-sombra. Os homens e mulheres-sombra são muito fortes e perigosos. Eles usam várias roupas de cores diferentes para caçar e fazer mal às pessoas hup. Uma dessas roupas tem cor de sombra, daí o seu nome. A gente-sombra causa doenças e pode matar e comer a carne e o espírito das pessoas humanas. Mas muitos deles são sábios, e conhecem cantos, mitos e benzimentos. Esta é a história do encontro de um homem hup com um homem-sombra chamado Way Naku.}

\chapter{}

No alto de uma árvore de\\
cucura, um homem-sombra\\
chamado Way Naku cantava\\
um \textit{caapivaiá}.


\textit{Batɨ̗b’ mah yup, Way Naku\\
batɨ̗b’ mah yup pö̗h yamah,\\
pɨ̗g tëganah.}


\chapter{}

Encantado com a música,\\
um homem hup aproximou-se,\\
sentou-se ao pé da\\
árvore e ficou escutando as\\
canções do homem-sombra. O\\
homem hup estava sentado,\\
pescando no igarapé\\
enquanto ouvia a música.

\textit{Pɨ̗g tëgan tɨh yam mɨ’ mah,\\
hup tɨh mɨ’ wɨ’ pemeh, tú.\\
Dë̖h máh hõ̖p käk pemep ɨhɨh.}

\chapter{}

Dizem que o Way Naku ficou\\
cantando e colhendo cucura\\
por muito tempo. O homem-sombra\\
ia de galho em galho\\
cantando e pegando os cachos\\
da fruta. Muitos dos cantos\\
que nós Hupd’äh cantamos nas\\
festas foram cantados pelo Way\\
Naku: o \textit{Canto do Cutia}, o \textit{Canto\\
Grande}, o \textit{Canto do Umarí}, o\\
\textit{Canto da Gente-Sombra}.

\textit{Bɨ̗ g! mah tih yamah. Sã̗p nowot\
pɨd, sã̗p nowot pɨd, pɨ̗g tëgët\\
tɨh noh k’ët kötöh, yamap ɨhɨh,\\
batɨ̗b’ih. \textit{Mèt Yam}, \textit{Yam Pög}, \textit{Pej\\
D’áp Yam}, \textit{Batɨ̗b’ Yam}, yám nihũ̗’\\
mah tɨ̗ h yamah!}

\chapter{}

O homem hup ficou\\
lá a noite toda e,\\
depois, o outro dia\\
inteiro. Escutou e\\
prendeu os cantos\\
do homem-sombra.

\textit{Hiwag yɨ’ɨy mah,\\
d’ú’ tɨh yam d’ö’öp\ldots{}}

\chapter{}

Quando o sol\\
começou a nascer,\\
o homem hup bateu\\
com o terçado no\\
tronco da árvore\\
de cucura.

\textit{Hiwag yɨ’ɨy, wag\\
hiyèt mah tɨh kɨt\\
hik’ëtayah, tɨh\\
të̖gët, pɨ̗ g tëgët.}

\chapter{}

Tac! Foi o barulhão\\
que fez. Assustado,\\
o Way Naku caiu da\\
árvore. \textit{Puffff}!\\
Com medo, o homem\\
hup foi embora\\
correndo, bem\\
rápido! \textit{Vrummm}!

\textit{Täk! nomɨ’, yuway\\
mah tɨh kädhiayah.\\
\textit{Pë’}! Yuway mah\\
húp to’oh kädham\\
yɨ’ayah! \textit{Mmmm’}!}

\chapter{}

Então, o Way Naku viu um\\
jacaré que estava deitado\\
na beira do rio. Pensou\\
que fosse o homem hup e\\
tentou pegá-lo. Fugindo,\\
o jacaré caiu na água,\\
\textit{tchibum}! O Way Naku,\\
que já estava na água,\\
foi atrás do jacaré. O\\
homem-sombra colocou a\\
mão na água para ver se\\
encontrava o jacaré.

De repente, o jacaré deu uma\\
mordida no braço do\\
homem-sombra.

\textit{Yɨt mah hàt yetníh. Yɨt\\
mah hàt noh tu’uh, \textit{tapuh}!\\
Yɨt mah tɨhɨt yɨ’ batɨ̗b’ noh\\
tu’ won kädd’öböh. Yɨt\\
mah hàtan tɨh d’ö’ yɨ’ɨh,\\
pëpë’ d’ö’ yɨ’ɨy.}

\textit{Yɨt mah tɨhan tɨh k’äç d’ö’\\
pög b’ayah, hàt b’ayah,\\
tɨnɨh mumùy súm,\\
batɨ̗b’anah.}

\chapter{}

E foi assim que\\
o homem hup\\
conseguiu fugir e\\
voltar para casa.

Esse foi o primeiro homem hup\\
que soube cantar os \textit{caapivaiá},\\
os cantos das festas.

\textit{Tɨhɨp húpup ham\\
yɨ’ay mah kah, yë\\
yɨ’ay mah.}

\textit{Yup ĩh mah yup, yám\\
d’ö’ hib’ahayah.}

\chapter{}

Ele ouviu o homem-sombra cantar,\\
aprendeu bem e\\
começou a ensinar\\
para seus filhos,\\
irmãos, cunhados.\\
E até hoje todos\\
cantam e dançam os\\
\textit{caapivaiá} que animam\\
nossas festas.

Foi isso o que ouvi de\\
meus avós.

\textit{Batɨ̗b’ tɨh yamawan\\
wɨ’yö’ay mah, yup húp\\
yamayah. Yup hɨd yam\\
tëg n’ɨhayah. Húpd’äh\\
hɨd yam tëgayah.}

\textit{Ya’ap meh yɨ’ ãh wɨ’ɨh.}


\chapter{Um canto do homem-sombra}

\begin{quote}
Outro galho de cucura\\
Outro galho de cucura\\
Eu jogo para baixo

Outro galho de cucura\\
Eu jogo para baixo\\
Colho o cacho de cucura\\
Eu jogo para baixo

\textit{Way Naku, yari nóóy mah\\
Marika, Way Naku}
\end{quote}

\chapter{Way Naku}

\begin{quote}
Sã̗p nowot pɨ̗ g\\
Sã̗p nowot pɨ̗ g\\
Ãh d’äräh hitëh

Sã̗p nowot pɨ̗ g\\
Ãh d’äräh hitëh\\
Öy d’ö’ yö’ pɨ̗ g\\
Ãh d’äräh hitëh

\textit{Way Naku, yari nóóy mah\\
Marika, Way Naku}
\end{quote}

\chapter{Glossário}


\textbf{Caapivaiá} Cantos alegres de festa.

\textbf{Cucura} Fruta da floresta amazônica que nasce em cachos
como uma uvas. O nome científico é \textit{Pouroma Cecropiaefolia}.

\textbf{Cutia} Animal mamífero e roedor, pequeno, que vive na floresta. O nome
científico da cutia é \textit{Dasyprocta Fuliginosa}.

\textbf{Gente sombra} A gente sombra, homens e mulheres sombra, são
muito fortes e perigosos. Geralmente, causam doenças, podem matar
e comer a carne e o espírito das pessoas humanas. Muitos deles
são sábios e conhecem cantos, mitos e benzimentos. A cor de
sombra, de onde vem seu nome, é a cor de uma das roupas que essas gentes
usam para caçar e fazer mal às pessoas hup.

\textbf{Homem sombra} Ser parecido com as pessoas
humanas, que vive na mata. Alguns, como Way Naku, vivem próximo a árvores frutíferas. Ele
é muito forte, sábio e poderoso. Ver \textbf{Gente­ sombra}.

\textbf{Terçado} Facão grande muito utilizado pelos Hupd’äh para cortar
galhos e árvores.

\textbf{Umari} Fruta da floresta amazônica de cor
amarela e de sabor um pouco amargo, muito apreciada pelos Hupd’äh.
Cresce numa árvore chamada umarizeiro, cujo o nome científico é \textit{Poraqueira Sp}.











