\chapter{Para ler as palavras hup}

Para a grafia dos termos da língua hup em geral, adotou"-se como
referência o dicionário de língua hup elaborado pelo linguista Henri
Ramirez, \emph{A Língua dos Hupd'äh do Alto Rio Negro} (Associação Saúde
Sem Limites, 2006). Seguindo Ramirez, mantém-se a acentuação
das vogais de acordo com a nasalidade (indicada por um til) e o tom
(indicado por um acento grave agudo ou grave).

Ramirez propõe que o alfabeto hup possui 25 letras: a, ä, b, ç, e, ë, g,
h, i, ɨ, j, k, m, n, o, ö, p, r, s, t, u, w, y, ' (oclusão
glotal). Destas, 16 são consoantes, 9 são vogais e 11 são consoantes
laringalizadas (b', d', r', j', g', m', n', w', y', s', k').

\bigskip

\begin{table}[ht!]
\parbox{.45\linewidth}{
\centering
\begin{tabular}{ccccc}
\hline
\multicolumn{5}{c}{\textsc{consoantes}}                                  \\ \hline
P & t                                               & s & k & ’ \\ \hline
B & \begin{tabular}[c]{@{}c@{}}d\\ (r)\end{tabular} & j & g &   \\ \hline
M & n                                               &   &   &   \\ \hline
  &                                                 & ç &   & h \\ \hline
W &                                                 & y &   &   \\ \hline
\end{tabular}
}
\hfill
\parbox{.45\linewidth}{
\centering
\begin{tabular}{ccc}
\hline
\multicolumn{3}{c}{\textsc{vogais}} \\ \hline
i       & ɨ       & U      \\ \hline
ë       & Ä       & ö      \\ \hline
e       & A       & o      \\ \hline
\end{tabular}
}
\end{table}

