\chapter{Para ler as palavras hup}

Para a grafia em geral dos termos da língua hup, adotou"-se como
referência o dicionário de língua hup elaborado pelo linguista Henri
Ramirez, \textit{A língua dos Hupd'äh do Alto Rio Negro} (Associação Saúde
Sem Limites, 2006). Seguindo Ramirez, mantém-se a acentuação
das vogais de acordo com a nasalidade --- indicada por um \textit{til} --- e o tom --- indicado por um acento grave agudo ou grave.

Ramirez propõe que o alfabeto hup possui 25 letras: \textit{a}, \textit{ä}, \textit{b}, \textit{ç}, \textit{e}, \textit{ë}, \textit{g},
\textit{h}, \textit{i}, \textit{ɨ}, \textit{j}, \textit{k}, \textit{m}, \textit{n}, \textit{o}, \textit{ö}, \textit{p}, \textit{r}, \textit{s}, \textit{t}, \textit{u}, \textit{w}, \textit{y} e \textit{'} (a oclusão
glotal). Destas, 16 são consoantes, 9 são vogais e 11 são consoantes
laringalizadas: \textit{b'}, \textit{d'}, \textit{r'}, \textit{j'}, \textit{g'}, \textit{m'}, \textit{n'}, \textit{w'}, \textit{y'}, \textit{s'}, \textit{k'}.

% \bigskip

% \begin{table}[ht!]
% \parbox{.45\linewidth}{
% \centering
% \begin{tabular}{ccccc}
% \hline
% \multicolumn{5}{c}{\textsc{consoantes}}                                  \\ \hline
% P & t                                               & s & k & ’ \\ \hline
% B & \begin{tabular}[c]{@{}c@{}}d\\ (r)\end{tabular} & j & g &   \\ \hline
% M & n                                               &   &   &   \\ \hline
%   &                                                 & ç &   & h \\ \hline
% W &                                                 & y &   &   \\ \hline
% \end{tabular}
% }
% \hfill
% \parbox{.45\linewidth}{
% \centering
% \begin{tabular}{ccc}
% \hline
% \multicolumn{3}{c}{\textsc{vogais}} \\ \hline
% i       & ɨ       & U      \\ \hline
% ë       & Ä       & ö      \\ \hline
% e       & A       & o      \\ \hline
% \end{tabular}
% }
% \end{table}

