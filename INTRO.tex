\chapter{Introdução}

Os Hupd’äh são um povo indígena que vive em aldeias espalhadas pela floresta Amazônica na região do Alto Rio Negro, no estado do Amazonas, na fronteira com a Colômbia.

A população total é cerca de 1500 pessoas, que moram em 35 aldeias diferentes. Antigamente, moravam de 15 a 50 pessoas nas aldeias Hupd’äh, mas atualmente as comunidades estão maiores e podem abrigar até 200 pessoas.

Os Hupd’äh costumam caçar com arco e flecha e com zarabatana. Pescam diferentes tipos de peixes nos igarapés, riachos que correm no interior da mata.

As mulheres dedicam-se diariamente às roças de maniva, um tipo de mandioca, mas também ajudam os homens na caça, na pesca e na coleta de frutas.

Quando viajam para visitar parentes em outras aldeias, os Hupd’äh andam por longos caminhos e acampam à beira do igarapés para descansar. Nestas viagens, é comum que, além de encontrarem animais como pacas, tamanduás, antas e cobras, os Hupd’äh encontrem outros tipos de gente, como a gente-sombra, gente-onça e a gente-árvore. Por isso, é preciso ter cuidado e respeitar essas outras gentes, para que elas não façam mal aos viajantes.

A língua hup é a primeira a ser falada pelas crianças Hupd’äh. Esta é uma língua tonal, muito
diferente do português. A língua hup é semelhante às línguas dos povos Yuhupdëh,
Dâw e Nadëb (Kuyawi) que também vivem em comunidades na região do Alto Rio Negro.

Desde 2001, os professores Hupd’äh começaram a participar de cursos para melhorar as
escolas de suas aldeias. Nesses cursos, os professores Hupd’äh tiveram a ajuda de linguistas e antropólogos para descrever sua língua, criar um modo de escrevê-la e compreender os princípios gramaticais.

Escutar as histórias dos anciões, gravá-las, escrevê-las e reproduzi-las permite que
crianças e adultos aprendam a ler e escrever na língua Hupd’äh.

Também durante este período, foi elaborado um dicionário Hupd’äh/português, em parceria
com pesquisadores de diferentes Universidades, além de cartilhas da língua hup.