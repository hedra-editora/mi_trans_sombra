\chapter*{}

Os Hupd'äh contam muitas histórias sobre pessoas que, andando pela mata, se
encontraram com seres da Gente-Sombra.

Os homens e mulheres-sombra são muito fortes e perigosos. Eles usam várias roupas de cores diferentes para caçar e fazer mal às pessoas hup. Uma dessas roupas tem cor de sombra,
daí o seu nome.

A Gente-Sombra causa doenças e pode matar e comer a carne e o espírito das pessoas humanas. Mas muitos deles são sábios, e conhecem cantos, mitos e benzimentos.

Esta é a história do encontro de um homem hup com um homem-sombra chamado Way Naku.

\chapter{Os cantos do homem-sombra}

No alto de uma árvore de
cucura, um homem-sombra
chamado Way Naku cantava
um caapivaiá.

Encantado com a música,
um homem hup aproximou-se, sentou-se ao pé da
árvore e ficou escutando as
canções do homem-sombra. O
homem hup estava sentado,
pescando no igarapé
enquanto ouvia a música.

Dizem que o Way Naku ficou
cantando e colhendo cucura
por muito tempo. O homem-sombra ia de galho em galho
cantando e pegando os cachos
da fruta. Muitos dos cantos
que nós Hupd’äh cantamos nas
festas foram cantados pelo Way
Naku: o Canto do Cutia, o Canto
Grande, o Canto do Umarí, o
Canto da Gente-Sombra.


O homem hup ficou
lá a noite toda e,
depois, o outro dia
inteiro. Escutou e
prendeu os cantos
do homem-sombra.


Quando o sol
começou a nascer,
o homem hup bateu
com o terçado no
tronco da árvore
de cucura.

Tac! Foi o barulhão
que fez. Assustado,
o Way Naku caiu da
árvore. Puffff!
Com medo, o homem
hup foi embora
correndo, bem
rápido! Vrummm!

Então, o Way Naku viu um
jacaré que estava deitado
na beira do rio. Pensou
que fosse o homem hup e
tentou pegá-lo. Fugindo,
o jacaré caiu na água,
tchibum! O Way Naku,
que já estava na água,
foi atrás do jacaré. O
homem-sombra colocou a
mão na água para ver se
encontrava o jacaré.

De repente, o jacaré deu
uma mordida no braço do
homem-sombra.

E foi assim que
o homem hup
conseguiu fugir e
voltar para casa.

Esse foi o primeiro
homem hup que
soube cantar
os caapivaiá , os
cantos das festas.

Ele ouviu o homem-sombra cantar,
aprendeu bem e
começou a ensinar
para seus filhos,
irmãos, cunhados.
E até hoje todos
cantam e dançam os
caapivaiá que animam
nossas festas.

Foi isso o que ouvi de
meus avós.


\chapter{Batɨ´b’ yám pɨnɨ´g}


Batíb’ mah yup, Way Naku
batíb’ mah yup pöh yamah,
pig tëganah.

Pig tëgan tih yam mi’ mah,
hup tih mi’ wi’ pemeh, tú.
Dëh máh hõp käk pemep ihih.

Bíg! mah tih yamah. Sãp nowot
pid, sãp nowot pid, pig tëgët
tih noh k’ët kötöh, yamap ihih,
batíb’ih. Mèt Yam, Yam Pög, Pej
D’áp Yam, Batíb’ Yam, yám nihu’
mah tíh yamah!

Hiwag yi’iy mah,
d’ú’ tih yam d’ö’öp\ldots{}

Hiwag yi’iy, wag
hiyèt mah tih kit
hik’ëtayah, tih
tëgët, pig tëgët.

Täk! nomi’, yuway
mah tih kädhiayah.
Pë’ ! Yuway mah
húp to’oh kädham
yi’ayah! Mmmm’ !

Yit mah hàt yetníh. Yit
mah hàt noh tu’uh, tapuh !
Yit mah tihit yi’ batíb’ noh
tu’ won kädd’öböh. Yit
mah hàtan tih d’ö’ yi’ih,
pëpë’ d’ö’ yi’iy.

Yit mah tihan tih k’äç
d’ö’ pög b’ayah, hàt
b’ayah, tinih mumùy súm,
batíb’anah.

Tihip húpup ham
yi’ay mah kah, yë
yi’ay mah.

Yup ih mah yup, yám
d’ö’ hib’ahayah.

Batíb’ tih yamawan
wi’yö’ay mah, yup húp
yamayah. Yup hid yam
tëg n’ihayah. Húpd’äh
hid yam tëgayah.

Ya’ap meh yi’ ãh wi’ih.


\chapter{Um canto do homem-sombra}

\begin{verse}
Outro galho de cucura\\
Outro galho de cucura\\
Eu jogo para baixo

Outro galho de cucura\\
Eu jogo para baixo\\
Colho o cacho de cucura\\
Eu jogo para baixo

Way Naku, yari nóóy mah\\
Marika, Way Naku
\end{verse}

\chapter{Way Naku}

\begin{verse}
Sãp nowot pig\\
Sãp nowot pig\\
Ãh d’äräh hitëh

Sãp nowot pig\\
Ãh d’äräh hitëh\\
Öy d’ö’ yö’ pig\\
Ãh d’äräh hitëh

Way Naku, yari nóóy mah\\
Marika, Way Naku
\end{verse}