\textbf{Os cantos do homem-sombra} \textls[10]{é a história do encontro de um Hup com um \textit{homem-sombra} chamado Way Naku. Ela faz parte de uma série de histórias dos Hupd'äh --- que vivem em aldeias espalhadas pela floresta amazônica, na região do Alto Rio Negro, fronteira entre o Brasil e a Colômbia --- sobre a \textit{gente-sombra}. Os \textit{homens} e \textit{mulheres-sombra} são muito perigosos e usam roupas coloridas, além de caçar e fazer mal aos Hup. Uma dessas roupas tem cor de sombra, daí seu nome. A \textit{gente-sombra} causa doenças e pode até matar. Eles comem a carne e o espírito dos humanos. Mas muitos deles são sábios e conhecem cantos, mitos e benzimentos.}

\textbf{Patience Epps} ​\textls[15]{é linguista, professora da Universidade do Texas \textsc{(ut)} e trabalha com os Hupd'äh desde 2000.}

\textbf{Danilo Paiva Ramos} \textls[10]{é antropólogo e professor adjunto do Departamento de Antropologia e Etnologia da Universidade Federal de Alfenas (\textsc{unifal--mg}). Além das pesquisas etnográficas, com ênfase em xamanismo, discurso, performance, vida ritual e territorialidade, é engajado em causas ligadas aos direitos indígenas. Assessora também o povo Hupd'äh para a construção de seus Planos de Gestão Territorial e Ambiental (\textsc{pgta}--Hup).}

\textbf{Anita Ekman} é artista visual, \textit{performer} e ilustradora que trabalha com as artes ameríndias e afro-brasileiras. Como especialista em arte indígena, trabalhou na formação da coleção \textit{Great Masters of Popular Art in Ibero-America}, do Banamex Cultural Fund.

\textls[25]{\textbf{Mundo Indígena} reúne materiais produzidos com pensadores de diferentes povos indígenas e pessoas que pesquisam, trabalham ou lutam pela garantia de seus direitos. Os livros foram feitos para serem utilizados pelas comunidades envolvidas na sua produção, e por isso uma parte significativa das obras é bilíngue. Esperamos divulgar a imensa diversidade linguística dos povos indígenas no Brasil, que compreende mais de 150 línguas pertencentes a mais de trinta famílias linguísticas.}


