\blankpage
\chapter{Glossário}


\textbf{Caapivaiá}\quad Cantos alegres de festa.\\
\textbf{Cucura}\quad Fruta da floresta amazônica que nasce em cachos
como uma uvas. O nome científico é \textit{Pouroma Cecropiaefolia}.\\
\textbf{Cutia}\quad Animal mamífero e roedor, pequeno, que vive na floresta. O nome
científico da cutia é \textit{Dasyprocta Fuliginosa}.\\
\textbf{Gente sombra}\quad A gente sombra, homens e mulheres sombra, são
muito fortes e perigosos. Geralmente, causam doenças, podem matar
e comer a carne e o espírito das pessoas humanas. Muitos deles
são sábios e conhecem cantos, mitos e benzimentos. A cor de
sombra, de onde vem seu nome, é a cor de uma das roupas que essas gentes
usam para caçar e fazer mal às pessoas hup.\\
\textbf{Homem sombra}\quad Ser parecido com as pessoas
humanas, que vive na mata. Alguns, como Way Naku, vivem próximo a árvores frutíferas. Ele
é muito forte, sábio e poderoso. Ver \textbf{Gente­ sombra}.\\
\textbf{Terçado}\quad Facão grande muito utilizado pelos Hupd’äh para cortar
galhos e árvores.\\
\textbf{Umari}\quad Fruta da floresta amazônica de cor
amarela e de sabor um pouco amargo, muito apreciada pelos Hupd’äh.
Cresce numa árvore chamada umarizeiro, cujo o nome científico é \textit{Poraqueira Sp}.


