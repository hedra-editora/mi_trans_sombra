\addtocontents{toc}{\medskip}
\chapter{Glossário}

\begin{itemize}
\item{\formular\textbf{Caapivaiá}}\quad Cantos alegres de festa.

\item{\formular\textbf{Cucura}}\quad {Fruta da Floresta Amazônica que nasce em cachos
como uvas. O nome científico é \textit{Pourouma cecropiaefolia}.}\looseness=-1

\item{\formular\textbf{Cutia}}\quad {Animal mamífero roedor e pequeno, vive na floresta. O nome
científico da cutia é \textit{Dasyprocta fuliginosa}.}

\item{\formular\textbf{Gente-sombra}}\quad A gente-sombra, homens e mulheres-sombra, são
muito fortes e perigosos. Geralmente causam doenças, podem matar
e comer a carne e o espírito das pessoas humanas. Muitos deles
são sábios e conhecem cantos, mitos e benzimentos. A cor de
sombra, de onde vem o nome, é a cor de uma das roupas que essas gentes
usam para caçar e fazer mal às pessoas hup.

\item{\formular\textbf{Homem-sombra}}\quad Ser parecido com as pessoas
humanas, vive na mata. Alguns, como Way Naku, vivem próximo de árvores frutíferas. Ele
é muito forte, sábio e poderoso. Ver \textit{Gente-sombra}.

\item{\formular\textbf{Terçado}}\quad Facão grande muito utilizado pelos Hupd’äh para cortar
galhos e árvores.

\item{\formular\textbf{Umari}}\quad {Fruta da Floresta Amazônica de cor
amarela e de sabor um pouco amargo, muito apreciada pelos Hupd’äh.
Cresce numa árvore chamada umarizeiro, cujo o nome científico é \textit{Poraqueira sp}.}
\end{itemize}

