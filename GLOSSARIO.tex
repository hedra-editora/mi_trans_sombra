\addtocontents{toc}{\medskip}
\chapter{Glossário}

\paragraph{Caapivaiá} Cantos alegres de festa.

\paragraph{Cucura} \textls[15]{Fruta da floresta amazônica que nasce em cachos
como uma uvas. O nome científico é \textit{Pouroma Cecropiaefolia}.}

\paragraph{Cutia} \textls[-15]{Animal mamífero e roedor, pequeno, que vive na floresta. O nome
científico da cutia é \textit{Dasyprocta Fuliginosa}.}

\paragraph{Gente sombra} A gente sombra, homens e mulheres sombra, são
muito fortes e perigosos. Geralmente, causam doenças, podem matar
e comer a carne e o espírito das pessoas humanas. Muitos deles
são sábios e conhecem cantos, mitos e benzimentos. A cor de
sombra, de onde vem seu nome, é a cor de uma das roupas que essas gentes
usam para caçar e fazer mal às pessoas hup.

\paragraph{Homem sombra} Ser parecido com as pessoas
humanas, que vive na mata. Alguns, como Way Naku, vivem próximo a árvores frutíferas. Ele
é muito forte, sábio e poderoso. Ver \textit{Gente sombra}.

\paragraph{Terçado} Facão grande muito utilizado pelos Hupd’äh para cortar
galhos e árvores.

\paragraph{Umari} \textls[15]{Fruta da floresta amazônica de cor
amarela e de sabor um pouco amargo, muito apreciada pelos Hupd’äh.
Cresce numa árvore chamada umarizeiro, cujo o nome científico é \textit{Poraqueira Sp}.}


