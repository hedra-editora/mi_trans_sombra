% Tamanhos
% \tiny
% \scriptsize
% \footnotesize
% \small 
% \normalsize
% \large 
% \Large 
% \LARGE 
% \huge
% \Huge

% Posicionamento
% \centering 
% \raggedright
% \raggedleft
% \vfill 
% \hfill 
% \vspace{Xcm}   % Colocar * caso esteja no começo de uma página. Ex: \vspace*{...}
% \hspace{Xcm}

% Estilo de página
% \thispagestyle{<<nosso>>}
% \thispagestyle{empty}
% \thispagestyle{plain}  (só número, sem cabeço)
% https://www.overleaf.com/learn/latex/Headers_and_footers

% Compilador que permite usar fonte de sistema: xelatex, lualatex
% Compilador que não permite usar fonte de sistema: latex, pdflatex

% Definindo fontes
% \setmainfont{Times New Roman}  % Todo o texto
% \newfontfamily\avenir{Avenir}  % Contexto

\begingroup\thispagestyle{empty}\vspace*{.05\textheight} 

              {\formular
              \huge
              \noindent
              \textbf{Os cantos do\\ homem-sombra}

              \vspace{0.6em}
              
              \Large
              \noindent
              \textit{Bat\i{}b’ yám p\I{}n\i{}g}
              }              
              \vspace{1cm}

              
              \small\noindent 
              Mário Pires, da aldeia Yuyu-Dëh, e Ponciano Socot,
              \vspace{-0.05cm}

              \noindent 
              da aldeia Tat-Dëh (\textit{narração})

              \medskip

              \noindent 
              Patience Epps e Danilo Paiva Ramos (\textit{organização})

              \medskip
              
              \noindent 
              Anita Ekman (\textit{ilustração})

              \vspace{0.5cm}

              \noindent
              1ª edição

              \vfill\noindent\includegraphics[width=0.3\textwidth]{LOGO_CDL.png}
              \break{} 
              \smallskip
              {\fontsize{30}{40}%\selectfont\minion
              \scriptsize\noindent São Paulo\quad\the\year}


\endgroup
\pagebreak
